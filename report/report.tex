\documentclass{tufte-handout}
\usepackage[utf8]{inputenc}
\usepackage{tikz}
\usepackage{amsmath}
\usepackage{float}
\usepackage{longtable}
\usepackage{adjustbox}

\usepackage{color}
\newcommand{\red}[1]{{\color{red} #1}}
\usepackage{booktabs}

\begin{document}
\section{Red Scare! Report}

by Andreas Tietgen (anti), Albert Rise Nielsen (albn) and Amalie Bøgild Sørensen (abso).

\subsection{Results}

The following table gives our results for all graphs of at least 500 vertices.

\medskip
\begin{longtable}{lrrrrrr}
\toprule
	Instance name & $n$ & A & F & M & N & S\\
	\midrule
	G-ex & 8 & true & 0 & ?! & 3 & ?!\\
	P3 & 3 & true & 1 & ?! & -1 & true\\
	common-1-100 & 100 & false & -1 & ?! & -1 & ?!\\
	common-1-20 & 20 & false & -1 & -1 & -1 & false\\
	common-1-250 & 250 & false & -1 & ?! & -1 & ?!\\
	common-1-50 & 50 & false & -1 & ?! & -1 & ?!\\
	common-1-500 & 500 & false & -1 & ?! & -1 & ?!\\
	common-2-100 & 100 & false & -1 & ?! & -1 & ?!\\
	common-2-20 & 20 & false & -1 & ?! & -1 & ?!\\
	common-2-250 & 250 & false & -1 & ?! & -1 & ?!\\
	common-2-50 & 50 & false & -1 & ?! & -1 & ?!\\
	common-2-500 & 500 & true & 1 & ?! & 4 & true\\
	dodecahedron & 20 & false & 1 & ?! & 1 & true\\
	gnm-10-15-0 & 10 & false & 0 & ?! & 1 & ?!\\
	gnm-10-15-1 & 10 & true & 0 & ?! & 1 & ?!\\
	gnm-10-20-0 & 10 & false & 0 & ?! & 3 & ?!\\
	gnm-10-20-1 & 10 & true & 1 & ?! & 3 & true\\
	grid-10-0 & 100 & false & 0 & ?! & 49 & ?!\\
	grid-10-1 & 100 & false & 0 & ?! & 29 & ?!\\
	grid-10-2 & 100 & true & 2 & ?! & -1 & true\\
	grid-5-0 & 25 & true & 0 & ?! & 14 & ?!\\
	grid-5-1 & 25 & true & 0 & ?! & 8 & ?!\\
	grid-5-2 & 25 & true & 1 & ?! & -1 & true\\
	increase-n10-1 & 10 & true & 1 & 1 & 1 & true\\
	increase-n10-2 & 10 & true & 1 & 2 & 1 & true\\
	increase-n10-3 & 10 & false & -1 & -1 & -1 & false\\
	increase-n100-1 & 100 & false & -1 & -1 & -1 & false\\
	increase-n100-2 & 100 & true & 1 & 12 & 1 & true\\
	increase-n100-3 & 100 & false & -1 & -1 & -1 & false\\
	increase-n20-1 & 20 & false & -1 & -1 & -1 & false\\
	increase-n20-2 & 20 & true & 0 & 5 & 1 & true\\
	increase-n20-3 & 20 & false & -1 & -1 & -1 & false\\
	increase-n50-1 & 50 & false & -1 & -1 & -1 & false\\
	increase-n50-2 & 50 & true & 1 & 4 & 1 & true\\
	increase-n50-3 & 50 & true & 1 & 3 & 1 & true\\
	increase-n500-1 & 500 & true & 1 & 15 & 1 & true\\
	increase-n500-2 & 500 & true & 1 & 16 & 1 & true\\
	increase-n500-3 & 500 & true & 1 & 15 & 1 & true\\
	increase-n8-1 & 8 & true & 1 & 1 & 1 & true\\
	increase-n8-2 & 8 & true & 0 & 1 & 1 & true\\
	increase-n8-3 & 8 & true & 0 & 1 & 1 & true\\
	rusty-1-17 & 17 & false & 0 & ?! & 10 & ?!\\
	ski-illustration & 36 & false & 0 & 1 & 8 & true\\
	ski-level10-1 & 79 & false & -1 & -1 & -1 & false\\
	ski-level10-2 & 76 & false & 1 & 4 & -1 & true\\
	ski-level10-3 & 77 & false & 4 & 7 & -1 & true\\
	ski-level20-1 & 252 & false & 3 & 10 & -1 & true\\
	ski-level20-2 & 253 & false & 9 & 13 & -1 & true\\
	ski-level20-3 & 254 & false & -1 & -1 & -1 & false\\
	ski-level3-1 & 16 & true & 0 & 3 & 5 & true\\
	ski-level3-2 & 14 & false & 0 & 1 & 5 & true\\
	ski-level3-3 & 15 & false & 0 & 1 & 5 & true\\
	ski-level5-1 & 29 & false & 1 & 3 & -1 & true\\
	ski-level5-2 & 26 & false & 1 & 2 & -1 & true\\
	ski-level5-3 & 27 & false & 0 & 0 & 7 & false\\
	smallworld-10-0 & 100 & false & 0 & ?! & 6 & ?!\\
	smallworld-10-1 & 100 & true & 0 & ?! & 8 & ?!\\
	smallworld-20-0 & 400 & false & 0 & ?! & 8 & ?!\\
	smallworld-20-1 & 400 & true & 0 & ?! & 14 & ?!\\
	smallworld-3-0 & 9 & false & 0 & ?! & 4 & ?!\\
	smallworld-3-1 & 9 & false & 1 & ?! & 2 & true\\
	wall-n-1 & 8 & false & 0 & ?! & 1 & ?!\\
	wall-n-10 & 80 & false & 0 & ?! & 1 & ?!\\
	wall-n-2 & 16 & false & 0 & ?! & 1 & ?!\\
	wall-n-3 & 24 & false & 0 & ?! & 1 & ?!\\
	wall-n-4 & 32 & false & 0 & ?! & 1 & ?!\\
	wall-p-1 & 8 & false & 0 & ?! & 1 & ?!\\
	wall-p-10 & 62 & false & 0 & ?! & 1 & ?!\\
	wall-p-2 & 14 & false & 0 & ?! & 1 & ?!\\
	wall-p-3 & 20 & false & 0 & ?! & 1 & ?!\\
	wall-p-4 & 26 & false & 0 & ?! & 1 & ?!\\
	wall-z-1 & 8 & false & 0 & ?! & 1 & ?!\\
	wall-z-10 & 71 & false & 0 & ?! & 1 & ?!\\
	wall-z-2 & 15 & false & 0 & ?! & 1 & ?!\\
	wall-z-3 & 22 & false & 0 & ?! & 1 & ?!\\
	wall-z-4 & 29 & false & 0 & ?! & 1 & ?!\\

	\bottomrule
\end{longtable}
\medskip

The columns are for the problems Alternate, Few, Many, None, and Some.
The table entries either give the answer, or contain `?' for those cases where we were unable to find a solution within reasonable time.
For those questions where there is a reason for our inability to find a good algorithm (because the problem is hard), we wrote `?!'.

For the complete table of all results, including time, see the tab-separated text file {\tt results.txt} or the appendix.

\subsection{Methods}

For problem A, I solved each instance $G$ by $\cdots$\footnote{Describe what you did.
  Use words like ``building a inverse anti-tree without self-loops where each vertex in $G$ is presented by a Strogatz--Wasserman shtump.
  I then performed a standard longest hash sorting using the algorithm of Bronf (Algorithm 5 in [1]).''
  Be neat, brief, and precise.}
The running time of this algorithm is $\cdot$, and my implementation spends $\cdots$ seconds on the instance $\cdots$ with  $n=\cdots$.

I solved problem $\cdots$ for all $\cdots$\footnote{For instance, “planar, bipartite”} graphs using $\cdots$.

I was unable to solve problem $\cdots$ except for the $\cdots$ instances.
This is because, in generality, this problem is $\cdots$. 
To see this, consider the following reduction from $\cdots$.
Let $\ldots$ 

I was also unable to solve $\cdots$ for $\cdots$, but I don’t know why.\footnote{Remove or expand as necessary.}

For problem F...

For problem M...

For problem N, we solved it by doing a BFS while avoiding to queue any red vertices unless it is the target node. BFS is guaranteed
to find the shortest path which is what we are trying to achieve here.
The runtime of BFS is O(V*E) where V are the number of vertices and E is the number of edges.

For problem S, we used the answer from problem M. Since problem M returns the maximum number of red nodes on a path from s to t we know that if this is > 0, we have
to return True and otherwise False for problem S. Thus, the Some problem reduces to the Many problem.
For that reason the runtime is the same as for problem M. 

\subsection{References}
\begin{description}
  \item[1.] \emph{APLgraphlib---A library for Basic Graph Algorithms in APL}, version 2.11, 2016, Iverson Project, {\tt github.com/iverson/APLgraphlib}.\sidenote{If you use references to code, books, or papers, be professional about it. Use whatever style you want, but be consistent.}

  \item[2.] A. Lovelace, \emph{Algorithms and Data Structures in Pascal}, Addison--Wesley 1881. 
\end{description}

\subsection{Appendix}
\textbf{All results}

\medskip
\begin{longtable}{lrrrrrr}
\toprule
	Instance name & $n$ & A & F & M & N & S\\
	\midrule
	G-ex & 8 & true & 0 & ?! & 3 & ?!\\
	P3 & 3 & true & 1 & ?! & -1 & true\\
	bht & 5757 & false & 0 & ?! & 6 & ?!\\
	common-1-100 & 100 & false & -1 & ?! & -1 & ?!\\
	common-1-1000 & 1000 & false & -1 & ?! & -1 & ?!\\
	common-1-1500 & 1500 & false & -1 & ?! & -1 & ?!\\
	common-1-20 & 20 & false & -1 & -1 & -1 & false\\
	common-1-2000 & 2000 & false & -1 & ?! & -1 & ?!\\
	common-1-250 & 250 & false & -1 & ?! & -1 & ?!\\
	common-1-2500 & 2500 & false & 1 & ?! & 6 & true\\
	common-1-3000 & 3000 & false & 1 & ?! & 6 & true\\
	common-1-3500 & 3500 & false & 1 & ?! & 6 & true\\
	common-1-4000 & 4000 & false & 1 & ?! & 6 & true\\
	common-1-4500 & 4500 & true & 1 & ?! & 6 & true\\
	common-1-50 & 50 & false & -1 & ?! & -1 & ?!\\
	common-1-500 & 500 & false & -1 & ?! & -1 & ?!\\
	common-1-5000 & 5000 & true & 1 & ?! & 6 & true\\
	common-1-5757 & 5757 & true & 1 & ?! & 6 & true\\
	common-2-100 & 100 & false & -1 & ?! & -1 & ?!\\
	common-2-1000 & 1000 & true & 1 & ?! & 4 & true\\
	common-2-1500 & 1500 & true & 1 & ?! & 4 & true\\
	common-2-20 & 20 & false & -1 & ?! & -1 & ?!\\
	common-2-2000 & 2000 & true & 1 & ?! & 4 & true\\
	common-2-250 & 250 & false & -1 & ?! & -1 & ?!\\
	common-2-2500 & 2500 & true & 1 & ?! & 4 & true\\
	common-2-3000 & 3000 & true & 1 & ?! & 4 & true\\
	common-2-3500 & 3500 & true & 1 & ?! & 4 & true\\
	common-2-4000 & 4000 & true & 1 & ?! & 4 & true\\
	common-2-4500 & 4500 & true & 1 & ?! & 4 & true\\
	common-2-50 & 50 & false & -1 & ?! & -1 & ?!\\
	common-2-500 & 500 & true & 1 & ?! & 4 & true\\
	common-2-5000 & 5000 & true & 1 & ?! & 4 & true\\
	common-2-5757 & 5757 & true & 1 & ?! & 4 & true\\
	dodecahedron & 20 & false & 1 & ?! & 1 & true\\
	gnm-10-15-0 & 10 & false & 0 & ?! & 1 & ?!\\
	gnm-10-15-1 & 10 & true & 0 & ?! & 1 & ?!\\
	gnm-10-20-0 & 10 & false & 0 & ?! & 3 & ?!\\
	gnm-10-20-1 & 10 & true & 1 & ?! & 3 & true\\
	gnm-1000-1500-0 & 1000 & false & 1 & ?! & -1 & true\\
	gnm-1000-1500-1 & 1000 & false & 2 & ?! & -1 & true\\
	gnm-1000-2000-0 & 1000 & false & 0 & ?! & 7 & ?!\\
	gnm-1000-2000-1 & 1000 & false & 1 & ?! & -1 & true\\
	gnm-2000-3000-0 & 2000 & false & 0 & ?! & 8 & ?!\\
	gnm-2000-3000-1 & 2000 & true & 2 & ?! & -1 & true\\
	gnm-2000-4000-0 & 2000 & false & 0 & ?! & 6 & ?!\\
	gnm-2000-4000-1 & 2000 & false & 0 & ?! & 5 & ?!\\
	gnm-3000-4500-0 & 3000 & false & 0 & ?! & 10 & ?!\\
	gnm-3000-4500-1 & 3000 & false & 2 & ?! & -1 & true\\
	gnm-3000-6000-0 & 3000 & false & 0 & ?! & 6 & ?!\\
	gnm-3000-6000-1 & 3000 & false & 1 & ?! & 6 & true\\
	gnm-4000-6000-0 & 4000 & false & 0 & ?! & 7 & ?!\\
	gnm-4000-6000-1 & 4000 & false & 0 & ?! & 15 & ?!\\
	gnm-4000-8000-0 & 4000 & false & 0 & ?! & 5 & ?!\\
	gnm-4000-8000-1 & 4000 & true & 1 & ?! & 6 & true\\
	gnm-5000-10000-0 & 5000 & false & 1 & ?! & 5 & true\\
	gnm-5000-10000-1 & 5000 & true & 0 & ?! & 5 & ?!\\
	gnm-5000-7500-0 & 5000 & false & -1 & ?! & -1 & ?!\\
	gnm-5000-7500-1 & 5000 & false & -1 & ?! & -1 & ?!\\
	grid-10-0 & 100 & false & 0 & ?! & 49 & ?!\\
	grid-10-1 & 100 & false & 0 & ?! & 29 & ?!\\
	grid-10-2 & 100 & true & 2 & ?! & -1 & true\\
	grid-25-0 & 625 & true & 0 & ?! & 324 & ?!\\
	grid-25-1 & 625 & true & 0 & ?! & 123 & ?!\\
	grid-25-2 & 625 & true & 5 & ?! & -1 & true\\
	grid-5-0 & 25 & true & 0 & ?! & 14 & ?!\\
	grid-5-1 & 25 & true & 0 & ?! & 8 & ?!\\
	grid-5-2 & 25 & true & 1 & ?! & -1 & true\\
	grid-50-0 & 2500 & false & 0 & ?! & 1249 & ?!\\
	grid-50-1 & 2500 & false & 0 & ?! & 521 & ?!\\
	grid-50-2 & 2500 & false & 10 & ?! & -1 & true\\
	increase-n10-1 & 10 & true & 1 & 1 & 1 & true\\
	increase-n10-2 & 10 & true & 1 & 2 & 1 & true\\
	increase-n10-3 & 10 & false & -1 & -1 & -1 & false\\
	increase-n100-1 & 100 & false & -1 & -1 & -1 & false\\
	increase-n100-2 & 100 & true & 1 & 12 & 1 & true\\
	increase-n100-3 & 100 & false & -1 & -1 & -1 & false\\
	increase-n20-1 & 20 & false & -1 & -1 & -1 & false\\
	increase-n20-2 & 20 & true & 0 & 5 & 1 & true\\
	increase-n20-3 & 20 & false & -1 & -1 & -1 & false\\
	increase-n50-1 & 50 & false & -1 & -1 & -1 & false\\
	increase-n50-2 & 50 & true & 1 & 4 & 1 & true\\
	increase-n50-3 & 50 & true & 1 & 3 & 1 & true\\
	increase-n500-1 & 500 & true & 1 & 15 & 1 & true\\
	increase-n500-2 & 500 & true & 1 & 16 & 1 & true\\
	increase-n500-3 & 500 & true & 1 & 15 & 1 & true\\
	increase-n8-1 & 8 & true & 1 & 1 & 1 & true\\
	increase-n8-2 & 8 & true & 0 & 1 & 1 & true\\
	increase-n8-3 & 8 & true & 0 & 1 & 1 & true\\
	rusty-1-17 & 17 & false & 0 & ?! & 10 & ?!\\
	rusty-1-2000 & 2000 & false & -1 & ?! & -1 & ?!\\
	rusty-1-2500 & 2500 & false & -1 & ?! & -1 & ?!\\
	rusty-1-3000 & 3000 & false & 0 & ?! & 14 & ?!\\
	rusty-1-3500 & 3500 & false & 0 & ?! & 14 & ?!\\
	rusty-1-4000 & 4000 & false & 0 & ?! & 13 & ?!\\
	rusty-1-4500 & 4500 & false & 0 & ?! & 7 & ?!\\
	rusty-1-5000 & 5000 & false & 0 & ?! & 7 & ?!\\
	rusty-1-5757 & 5757 & false & 0 & ?! & 7 & ?!\\
	rusty-2-2000 & 2000 & false & 0 & ?! & 5 & ?!\\
	rusty-2-2500 & 2500 & false & 0 & ?! & 4 & ?!\\
	rusty-2-3000 & 3000 & false & 0 & ?! & 4 & ?!\\
	rusty-2-3500 & 3500 & false & 0 & ?! & 4 & ?!\\
	rusty-2-4000 & 4000 & false & 0 & ?! & 4 & ?!\\
	rusty-2-4500 & 4500 & false & 0 & ?! & 4 & ?!\\
	rusty-2-5000 & 5000 & false & 0 & ?! & 4 & ?!\\
	rusty-2-5757 & 5757 & false & 0 & ?! & 4 & ?!\\
	ski-illustration & 36 & false & 0 & 1 & 8 & true\\
	ski-level10-1 & 79 & false & -1 & -1 & -1 & false\\
	ski-level10-2 & 76 & false & 1 & 4 & -1 & true\\
	ski-level10-3 & 77 & false & 4 & 7 & -1 & true\\
	ski-level20-1 & 252 & false & 3 & 10 & -1 & true\\
	ski-level20-2 & 253 & false & 9 & 13 & -1 & true\\
	ski-level20-3 & 254 & false & -1 & -1 & -1 & false\\
	ski-level3-1 & 16 & true & 0 & 3 & 5 & true\\
	ski-level3-2 & 14 & false & 0 & 1 & 5 & true\\
	ski-level3-3 & 15 & false & 0 & 1 & 5 & true\\
	ski-level5-1 & 29 & false & 1 & 3 & -1 & true\\
	ski-level5-2 & 26 & false & 1 & 2 & -1 & true\\
	ski-level5-3 & 27 & false & 0 & 0 & 7 & false\\
	smallworld-10-0 & 100 & false & 0 & ?! & 6 & ?!\\
	smallworld-10-1 & 100 & true & 0 & ?! & 8 & ?!\\
	smallworld-20-0 & 400 & false & 0 & ?! & 8 & ?!\\
	smallworld-20-1 & 400 & true & 0 & ?! & 14 & ?!\\
	smallworld-3-0 & 9 & false & 0 & ?! & 4 & ?!\\
	smallworld-3-1 & 9 & false & 1 & ?! & 2 & true\\
	smallworld-30-0 & 900 & false & 0 & ?! & 9 & ?!\\
	smallworld-30-1 & 900 & true & 0 & ?! & 11 & ?!\\
	smallworld-40-0 & 1600 & false & 0 & ?! & 8 & ?!\\
	smallworld-40-1 & 1600 & true & 0 & ?! & 13 & ?!\\
	smallworld-50-0 & 2500 & false & 0 & ?! & 3 & ?!\\
	smallworld-50-1 & 2500 & true & 1 & ?! & -1 & true\\
	wall-n-1 & 8 & false & 0 & ?! & 1 & ?!\\
	wall-n-10 & 80 & false & 0 & ?! & 1 & ?!\\
	wall-n-100 & 800 & false & 0 & ?! & 1 & ?!\\
	wall-n-1000 & 8000 & false & 0 & ?! & 1 & ?!\\
	wall-n-10000 & 80000 & false & 0 & ?! & 1 & ?!\\
	wall-n-2 & 16 & false & 0 & ?! & 1 & ?!\\
	wall-n-3 & 24 & false & 0 & ?! & 1 & ?!\\
	wall-n-4 & 32 & false & 0 & ?! & 1 & ?!\\
	wall-p-1 & 8 & false & 0 & ?! & 1 & ?!\\
	wall-p-10 & 62 & false & 0 & ?! & 1 & ?!\\
	wall-p-100 & 602 & false & 0 & ?! & 1 & ?!\\
	wall-p-1000 & 6002 & false & 0 & ?! & 1 & ?!\\
	wall-p-10000 & 60002 & false & 0 & ?! & 1 & ?!\\
	wall-p-2 & 14 & false & 0 & ?! & 1 & ?!\\
	wall-p-3 & 20 & false & 0 & ?! & 1 & ?!\\
	wall-p-4 & 26 & false & 0 & ?! & 1 & ?!\\
	wall-z-1 & 8 & false & 0 & ?! & 1 & ?!\\
	wall-z-10 & 71 & false & 0 & ?! & 1 & ?!\\
	wall-z-100 & 701 & false & 0 & ?! & 1 & ?!\\
	wall-z-1000 & 7001 & false & 0 & ?! & 1 & ?!\\
	wall-z-10000 & 70001 & false & 0 & ?! & 1 & ?!\\
	wall-z-2 & 15 & false & 0 & ?! & 1 & ?!\\
	wall-z-3 & 22 & false & 0 & ?! & 1 & ?!\\
	wall-z-4 & 29 & false & 0 & ?! & 1 & ?!\\

	\bottomrule
\end{longtable}
\medskip


\end{document}

